% Slides for 2025-03-31
% To create a slide, use the following:
% \begin{frame}{TITLE}
%     BODY
% \end{frame}

% To create a slide with a bullet list, use the following:
% \begin{frame}{TITLE}
%     \begin{itemize}
%         \item ITEM 1
%         \item ITEM 2
%     \end{itemize}    
% \end{frame}

% To create a slide with numbered list, use the following:
% \begin{frame}{TITLE}
%     \begin{enumerate}
%         \item ITEM 1
%         \item ITEM 2
%     \end{enumerate}
% \end{frame}

% To create a slide with a graphic:
% 1. Add the graphic to this folder (named picture.png)
% 2. Use the following:
% \begin{frame}{TITLE}
%     \centering
%     \includegraphics[height=0.7\textheight,width=0.7\textwidth,keepaspectratio]{picture.png}
% \end{frame}

% To create a slide with two columns, use the following:
% \begin{frame}{TITLE}
%     \begin{columns}
%         \begin{column}{0.5\textwidth}
%             COLUMN 1 BODY
%         \end{column}
%         \begin{column}{0.5\textwidth}
%             COLUMN 2 BODY
%         \end{column}
%     \end{columns}
% \end{frame}


\begin{frame}{Direction for this quarter}
    \begin{itemize}
        \item Create a modular pipeline which inputs a rough point cloud, refines it, creates a mesh with acceptable detail and gives a VR-compatible Unity project.
        \item Should be accessible to an archeologist without technical expertise.
    \end{itemize}    
\end{frame}

\begin{frame}{Progress so far}
    \begin{itemize}
        \item Set up a preliminary Docker build of CloudCompare using Ubuntu 22.04 as the base image.
        \item Wrote a powershell script to run the docker image.
        \item Successfully performed spatial subsampling on some test data with CloudCompare without GUI.
    \end{itemize}    
\end{frame}

\begin{frame}{Further}
    \begin{itemize}
        \item Incorporate more CloudCompare operations and test different data for finding the optimal parameters.
        \item Evaluate whether to continue using CloudCompare or switch to Open3D’s TSDF for mesh construction.
        \item Begin integrating the output mesh into Unity with basic VR compatibility.
    \end{itemize}    
\end{frame}
