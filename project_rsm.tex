% Slides for 2024-07-09
% To create a slide, use the following:
\begin{frame}{LAST WEEK}
    \begin{enumerate}
        \item Paper reading
        \begin{itemize}
            \item{Neural Network Quantization for Efficient Inference: A Survey}
            % a survey abt NN quantization
            % which is one way to reduce the size and complexity and achieve efficient inference
            % so that is easier to deploy on the mobile device
            % learned the main trade-off between precision and accuracy, and several quantization algorithms
            \item{Deep Compression: Compressing Deep Neural Networks With Pruning, Trained Quantization and Huffman Coding}
            % combination of pruning, quantization, and Huffman encoding 
            % achieve 35x to 49x reduction with the same accuracy 
        \end{itemize}
    \end{enumerate}
\end{frame}

\begin{frame}{LAST WEEK}
    \begin{enumerate}
        \item Assignment for Neural Network assignment
        \begin{itemize}
            \item{T:} Done with dataLoader & start with understanding of algorithm of NN
            % reading car images 
            % and store them as a Python dictionary with car-type keys 
            % and corresponding values as a list of NumPy arrays for the images associated with the car type.
            \item{W:} Done with separate data into train/val/test
            \item{Th+F:} Figure out PCA and start with logistic regression
            % PCA is Principal Components Analysis, which performs dimensionality reduction on the images
            % where logistic regression is the way to train data with two and only two groups
        \end{itemize}
    \end{enumerate}
\end{frame}

% To create a slide with a bullet list, use the following:
% \begin{frame}{TITLE}
%     \begin{itemize}
%         \item ITEM 1
%         \item ITEM 2
%     \end{itemize}    
% \end{frame}

% To create a slide with numbered list, use the following:
% \begin{frame}{TITLE}
%     \begin{enumerate}
%         \item ITEM 1
%         \item ITEM 2
%     \end{enumerate}
% \end{frame}

% To create a slide with a graphic:
% 1. Add the graphic to this folder (named picture.png)
% 2. Use the following:
% \begin{frame}{TITLE}
%     \centering
%     \includegraphics[height=0.7\textheight,width=0.7\textwidth,keepaspectratio]{picture.png}
% \end{frame}

% To create a slide with two columns, use the following:
% \begin{frame}{TITLE}
%     \begin{columns}
%         \begin{column}{0.5\textwidth}
%             COLUMN 1 BODY
%         \end{column}
%         \begin{column}{0.5\textwidth}
%             COLUMN 2 BODY
%         \end{column}
%     \end{columns}
% \end{frame}
